%use command pdflatex filename.tex
\documentclass{article}
%\usepackage[noend]{algpseudocode}
%\usepackage{parskip}
% \usepackage{tikz, changepage, amssymb}
% \usetikzlibrary{automata,positioning,arrows}
\usepackage{amsmath, amsthm}
\usepackage{listings}
\usepackage{xcolor}
\usepackage{tikz}
\usepackage{array}
\usepackage{url}

\lstset{
  language=python,
  basicstyle=\ttfamily,
  mathescape,
  morecomment=[l][\color{olive}]{\#},
  breaklines=true,
  postbreak=\space
}
\title{DATASCI 2G03 Project Proposal \\\large
Elastic Collisions between Contained Particles}
\author{Anthony Hunt}

\begin{document}
\maketitle

\section*{Introduction}
Simple 2D video games like pong, brick breaker, or even 8 ball pool have a great deal of both physical and computational concerns. In particular, this project aims to simulate the motion of moving particles and the collision interactions between them.
\\
The main problems we will be solving are:

\begin{itemize}
    \item Motion of several (round) particles with parametric starting configurations
    \item Elastic collisions between particles and a bounding box
\end{itemize}

The first part of this project will consist of modelling the interaction between many particles in a containing box the hopes of achieving a 2D Maxwell–Boltzmann distribution of velocities. Then, we will extend these interactions with the addition of charged particles to simulate non-constant acceleration.


\section*{Equations}
The ODEs and other equations used to simulate this problem are listed below. The goal of this simulation is to track the position of objects (represented by $s$) over some length of time $t$. Each particle will have some mass $m$ and starting velocity $v$.

Velocity ($v$) of one particle:
$$v = \frac{ds}{dt}$$

Acceleration ($a$) of one particle:
$$a = \frac{dv}{dt} = \frac{d^2s}{dt^2}$$

Momentum of one particle:
$$p = mv$$

Kinetic energy of one particle:
$$E = \frac{1}{2}mv^2$$

Newton's second law:
$$F = ma$$

The below equations are concerned with calculating the vectors of velocity resulting from an elastic collision. Let $v_1$ represent the first particle's velocity and $v_2$ for the second particle's velocity. Additionally, let $v'$ represent the velocity after the collision and $v$ for the velocities prior to the collision.

$$
\begin{aligned}
\mathbf {v} '_{1}&=\mathbf {v} _{1}-{\frac {2m*{2}}{m*{1}+m*{2}}}\ {\frac {\langle \mathbf {v} *{1}-\mathbf {v} _{2},\,\mathbf {x} _{1}-\mathbf {x} _{2}\rangle }{\|\mathbf {x} _{1}-\mathbf {x} _{2}\|^{2}}}\ (\mathbf {x} _{1}-\mathbf {x} _{2})\\
\mathbf {v} '_{2}&=\mathbf {v} _{2}-{\frac {2m_{1}}{m*{1}+m*{2}}}\ {\frac {\langle \mathbf {v} _{2}-\mathbf {v} _{1},\,\mathbf {x} _{2}-\mathbf {x} _{1}\rangle }{\|\mathbf {x} _{2}-\mathbf {x} _{1}\|^{2}}}\ (\mathbf {x} _{2}-\mathbf {x} _{1})
\end{aligned}
$$

Coulomb's law: Force of charged particles against one another:
$$F_{charge} = \frac{kq_1q_2}{r^2}$$

\section*{Configuration Parameters}
\begin{itemize}
    \item Starting position $s$ of each particle
    \item Initial velocity $v$ of each particle
    \item Mass $m$ of each particle
    \item Radius of each particle $r$
    \item Charge of different particles $q$
\end{itemize}

\section*{Interesting Properties}
\begin{itemize}
    \item Number of times a particle is hit before stopping the simulation
    \item Speed of particles over time (momentum of all particles should remain constant) position of particles on a graph
    \item Distribution of particle speeds (Maxwell–Boltzmann distribution)
    \item Unique path of each particle
\end{itemize}

\section*{References}
\begin{itemize}
    \item Equations of motion: \url{https://en.wikipedia.org/wiki/Equations_of_motion}
    \item Elastic collisions: \url{https://en.wikipedia.org/wiki/Elastic_collision}
    \item Maxwell-Boltzmann distribution: \url{https://en.wikipedia.org/wiki/Maxwell-Boltzmann_distribution}
\end{itemize}


\end{document}
